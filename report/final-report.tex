%File: formatting-instructions-latex-2025.tex
%release 2025.0
\documentclass[letterpaper]{article} % DO NOT CHANGE THIS
\usepackage{aaai25}  % DO NOT CHANGE THIS
\usepackage{times}  % DO NOT CHANGE THIS
\usepackage{helvet}  % DO NOT CHANGE THIS
\usepackage{courier}  % DO NOT CHANGE THIS
\usepackage[hyphens]{url}  % DO NOT CHANGE THIS
\usepackage{graphicx} % DO NOT CHANGE THIS
\urlstyle{rm} % DO NOT CHANGE THIS
\def\UrlFont{\rm}  % DO NOT CHANGE THIS
\usepackage{natbib}  % DO NOT CHANGE THIS AND DO NOT ADD ANY OPTIONS TO IT
\usepackage{caption} % DO NOT CHANGE THIS AND DO NOT ADD ANY OPTIONS TO IT
\frenchspacing  % DO NOT CHANGE THIS
\setlength{\pdfpagewidth}{8.5in}  % DO NOT CHANGE THIS
\setlength{\pdfpageheight}{11in}  % DO NOT CHANGE THIS
%
% These are recommended to typeset algorithms but not required. See the subsubsection on algorithms. Remove them if you don't have algorithms in your paper.
\usepackage{algorithm}
\usepackage{algorithmic}

\usepackage{tabularray}

%
% These are are recommended to typeset listings but not required. See the subsubsection on listing. Remove this block if you don't have listings in your paper.
\usepackage{newfloat}
\usepackage{listings}
\DeclareCaptionStyle{ruled}{labelfont=normalfont,labelsep=colon,strut=off} % DO NOT CHANGE THIS
\lstset{%
	basicstyle={\footnotesize\ttfamily},% footnotesize acceptable for monospace
	numbers=left,numberstyle=\footnotesize,xleftmargin=2em,% show line numbers, remove this entire line if you don't want the numbers.
	aboveskip=0pt,belowskip=0pt,%
	showstringspaces=false,tabsize=2,breaklines=true}
\floatstyle{ruled}
\newfloat{listing}{tb}{lst}{}
\floatname{listing}{Listing}
%
% Keep the \pdfinfo as shown here. There's no need
% for you to add the /Title and /Author tags.
\pdfinfo{
/TemplateVersion (2025.1)
}

% DISALLOWED PACKAGES
% \usepackage{authblk} -- This package is specifically forbidden
% \usepackage{balance} -- This package is specifically forbidden
% \usepackage{color (if used in text)
% \usepackage{CJK} -- This package is specifically forbidden
% \usepackage{float} -- This package is specifically forbidden
% \usepackage{flushend} -- This package is specifically forbidden
% \usepackage{fontenc} -- This package is specifically forbidden
% \usepackage{fullpage} -- This package is specifically forbidden
% \usepackage{geometry} -- This package is specifically forbidden
% \usepackage{grffile} -- This package is specifically forbidden
% \usepackage{hyperref} -- This package is specifically forbidden
% \usepackage{navigator} -- This package is specifically forbidden
% (or any other package that embeds links such as navigator or hyperref)
% \indentfirst} -- This package is specifically forbidden
% \layout} -- This package is specifically forbidden
% \multicol} -- This package is specifically forbidden
% \nameref} -- This package is specifically forbidden
% \usepackage{savetrees} -- This package is specifically forbidden
% \usepackage{setspace} -- This package is specifically forbidden
% \usepackage{stfloats} -- This package is specifically forbidden
% \usepackage{tabu} -- This package is specifically forbidden
% \usepackage{titlesec} -- This package is specifically forbidden
% \usepackage{tocbibind} -- This package is specifically forbidden
% \usepackage{ulem} -- This package is specifically forbidden
% \usepackage{wrapfig} -- This package is specifically forbidden
% DISALLOWED COMMANDS
% \nocopyright -- Your paper will not be published if you use this command
% \addtolength -- This command may not be used
% \balance -- This command may not be used
% \baselinestretch -- Your paper will not be published if you use this command
% \clearpage -- No page breaks of any kind may be used for the final version of your paper
% \columnsep -- This command may not be used
% \newpage -- No page breaks of any kind may be used for the final version of your paper
% \pagebreak -- No page breaks of any kind may be used for the final version of your paperr
% \pagestyle -- This command may not be used
% \tiny -- This is not an acceptable font size.
% \vspace{- -- No negative value may be used in proximity of a caption, figure, table, section, subsection, subsubsection, or reference
% \vskip{- -- No negative value may be used to alter spacing above or below a caption, figure, table, section, subsection, subsubsection, or reference

\setcounter{secnumdepth}{2} %May be changed to 1 or 2 if section numbers are desired.

% The file aaai25.sty is the style file for AAAI Press
% proceedings, working notes, and technical reports.
%

\usepackage{hyperref}
\usepackage{xspace}

\newcommand{\tinyurl}[1]{\begin{scriptsize}\url{#1}\end{scriptsize}}

\usepackage{xcolor}
\newcommand{\bu}[1]{\textbf{\underline{#1}}}
\newcommand{\rbu}[1]{{\color{red}\textbf{\underline{#1}}}}
\newcommand{\gap}{\> \>}




%basic planning models
\newcommand{\action}[2]{\ensuremath{\textnormal{#1}(\textnormal{#2})}\xspace}
\newcommand{\operator}[2]{\ensuremath{\textnormal{#1}(\textnormal{#2})}\xspace}
\newcommand{\task}[2]{\ensuremath{\textnormal{#1}}(#2)\xspace}
\newcommand{\sv}[3]{\ensuremath{\textnormal{#1}(\textnormal{#2})}=\textnormal{#3}\xspace}
\newcommand{\svdef}[3]{\ensuremath{\textnormal{#1}(\textnormal{#2})\in \textnormal{#3}}\xspace}
\newcommand{\svnoparam}[2]{\ensuremath{\textnormal{#1}}=\textnormal{#2}\xspace}
\newcommand{\sr}[2]{\ensuremath{\textnormal{#1}(\textnormal{#2})}\xspace}
\newcommand{\srdef}[2]{\ensuremath{\textnormal{#1} \subseteq #2}\xspace}
\newcommand{\predicate}[1]{\ensuremath{\textnormal{#1}}\xspace}
\newcommand{\pred}[1]{\predicate{#1}}

\newcommand{\method}[1]{\ensuremath{\textnormal{#1}}\xspace}
\newcommand{\methodfull}[2]{\ensuremath{\textnormal{#1}(#2)}\xspace}
\newcommand{\name}[1]{\ensuremath{\textnormal{#1}}\xspace}
\newcommand{\codecomment}[1]{{\small {\em #1}}}

%heuristics
\newcommand{\hff}{\ensuremath{h^{\textnormal{FF}}}\xspace}
\newcommand{\hrl}{\ensuremath{h^{\textnormal{RL}}}\xspace}


%algorithms
\newcommand{\alg}[1]{{\fontfamily{qag}\selectfont \textit{#1}}\xspace}
\newcommand{\Simulate}{\alg{Simulate}}
\newcommand{\Lookahead}{\alg{Lookahead}}
\newcommand{\RunLookahead}{\alg{Run-Lookahead}}
\newcommand{\RunLazyLookahead}{\alg{Run-Lazy-Lookahead}}
\newcommand{\Frontier}{\alg{Frontier}}
\newcommand{\TOHTNAct}{\alg{TO-HTN-Act}}
\newcommand{\HTNRunLookahead}{\alg{HTN-Run-Lookahead}}
\newcommand{\HTNRunLazyLookahead}{\alg{HTN-Run-Lazy-Lookahead}}
\newcommand{\TOHTNForward}{\alg{TO-HTN-Forward}}
\newcommand{\valueiteration}{\alg{VI}}
\newcommand{\policyiteration}{\alg{PI}}
\newcommand{\bellmanupdate}{\alg{Bellman-Update}}
\newcommand{\bfs}{\alg{BFS}}
\newcommand{\dfs}{\alg{DFS}}
\newcommand{\gbfs}{\alg{GBFS}}
\newcommand{\astar}{\alg{A*}}
\newcommand{\idastar}{\alg{IDA*}}
\newcommand{\runpolicy}{\alg{Run-Policy}}
\newcommand{\laostar}{\alg{LAO*}}
\newcommand{\laostarupdate}{\alg{LAO*-Update}}
\newcommand{\aostar}{\alg{AO*}}
\newcommand{\laoupdate}{\alg{LAO-Update}}
\newcommand{\envelope}{\alg{Envelope}}
\newcommand{\Envelope}[1]{\envelope \ensuremath{= \{#1\}}}
\newcommand{\fringe}{\alg{Fringe}}
\newcommand{\Fringe}[1]{\fringe \ensuremath{ = \{#1\}}}
\newcommand{\leaves}{\alg{Leaves}}
\newcommand{\Leaves}[1]{\leaves \ensuremath{ = \{#1\}}}

\newcommand{\frontier}{\alg{Frontier}}
\newcommand{\select}{\alg{Select}}
\newcommand{\uct}{\alg{UCT}}



% Title

% Your title must be in mixed case, not sentence case.
% That means all verbs (including short verbs like be, is, using,and go),
% nouns, adverbs, adjectives should be capitalized, including both words in hyphenated terms, while
% articles, conjunctions, and prepositions are lower case unless they
% directly follow a colon or long dash
\title{Rideshare Pilots \\ \large Assisted Matching via Planning and Acting in Simulated Environments}
\author{
    Addison Hanrattie\\
    \href{https://github.com/supersimple33/rideshare-pilots}{Project link}
}
\affiliations{
    University of Maryland\\
    ahanratt@umd.edu
}

\nocopyright
\usepackage{tabularray}  % for the assessment table

\begin{document}

\maketitle


\section*{Submission Checklist}

\begin{small}
    
\begin{tabular}{|r|p{6.5cm}|}
\hline
Done &  Item\\
\hline
    \underline{~~X~~}  & Example completed task. \\
\multicolumn{2}{|c|}{}\\
\hline
\multicolumn{2}{|c|}{Prepare Your Code}\\
\hline
    \underline{~~~~}  & Push your code to a private GitHub repository.\\
                      & (If your results are not too large, please include your results too.)\\
    \underline{~~~~}  & Be sure your code has all planning models you used in your project.\\
    \underline{~~~~}  & Write a README.md that describes where things are in the repository.\\
    
    \underline{~~~~}  & \href{https://docs.github.com/en/account-and-profile/setting-up-and-managing-your-personal-account-on-github/managing-access-to-your-personal-repositories/inviting-collaborators-to-a-personal-repository}{Invite}  makro@umd.edu  and dhchan@cs.umd.edu as collaborators to the repository. \\
\multicolumn{2}{|c|}{}\\
\hline
\multicolumn{2}{|c|}{Fill the content (recommended order)}\\
\hline
    \underline{~~~~}  & Add your results and evidence \\
    \underline{~~~~}  & Describe your evaluation plan \\
    \underline{~~~~}  & Complete the general discussion of Seciton 5.1 \\
    \underline{~~~~}  & Fill in the Approach section\\
    \underline{~~~~}  & Fill in the background as needed to explain results and approach \\
    \underline{~~~~}  & Write the Introduction \\

\multicolumn{2}{|c|}{}\\
\hline
\multicolumn{2}{|c|}{Complete the self assessment (after 12/2)}\\
\hline
    \underline{~~~~}  & Copy the \texttt{grading-template.tex} from Piazza into your project. \\
    \underline{~~~~}  & Complete the self assessment. \\
    \underline{~~~~}  & Sign the pledge. \\
    
\multicolumn{2}{|c|}{}\\
\hline
\multicolumn{2}{|c|}{Submit your report (after 12/2)}\\
\hline
    \underline{~~~~}  & Create a PDF of this document\\
    \underline{~~~~}  & Submit the PDF to Gradescope, being sure to select the first page for the first question.\\
    \underline{~~~~}  & Email your PDF to makro@umd.edu.\\
    
%\multicolumn{2}{|c|}{}\\
%\hline
%\multicolumn{2}{|c|}{New Section}\\
%\hline
%    \underline{~~~~}  & \\
%    \underline{~~~~}  & $\cdots$ \\
\hline
\end{tabular}
\end{small}



\section*{Assessment}
\begin{small}
    
Your report will be distinguished by the following criteria:
\begin{itemize}
    \item Approach:
        \begin{itemize}
            \item Clearly states the technical approach of the work in a self-contained way.
            \item The acting environment is clearly explained; stronger reports will include an example figure. If the environment started as a Gym environment but was modified, this is clear.
            \item Includes a planning and acting component.  For the acting, simulation is fine
            \item Stronger reports will leverage or manipulate the integration in an interesting way.
        \end{itemize}
    
    \item Evaluation:
        \begin{itemize}
            \item Clear exposition of claims, questions, variables and protocol.
            \item Evidence to support the claims in the form of a table or plot and no table or plot has font smaller than \textbackslash scriptsize (about 6pt font).
            \item Plots have axes that are clearly labeled and their captions state their intended meaning.
            \item All plots and tables in Section~\ref{sec:figures-and-tables} are referenced in the main portion of the paper; supplemental plots that are not referenced should be placed in the appendix.  
            \item A discussion of findings and how evidence relates to the claims
            \item Stronger reports will have a baseline approach; while not required, there is already evidence of this in some reports.
            \item It is certainly not required for this report, but you are welcome to include tests of statistical significance if this is something you know how to do.
            \item The strongest reports will demonstrate what I call second-order thinking.  This is where you conjecture a possible reason for the results you saw.  It is even stronger if you run one or more experiments to verify this conjecture or provide analytical results showing why this is the case.
        \end{itemize}

    \item Scope and Writing
        \begin{itemize}
            \item Paper tells an interesting or noteworthy story rather than just a chronology of experiments. (Exhibits first-order thinking)
            \item Written content is 2-4 pages, excluding floats (figures and tables), and floats are correctly placed in Section~\ref{sec:figures-and-tables} to make this assessment easy!
            \item Writing is easy to understand
            \item Reader is not left "wondering".  Consider yourself a teacher of your project. What would your `student' need to know to understand the content?
            \item Grammar is mostly sound; no obvious typos, misspelled words, sentence fragments, etc.
            \item Citations are correct and consistent; URLs are fine for hyperlinks, but, generally, books and articles should use a bibtex entry. 
            
        \end{itemize}


%    \item New section
%        \begin{itemize}
%            \item Clear and concise claims
%        \end{itemize}

        
\end{itemize}

\end{small}
\clearpage

\onecolumn 
% ========================================================================
% ========================================================================
% ========================================================================
Student Assessment: Please complete the orange boxes, replacing text within \textless .. \textgreater.  For example, in the first entry you would replace "\textless replace:\{ 1, 2, 3 \} \textgreater" with "1" if you did a type 1 project. I suggest you change these one at a time and recomplile each time to make sure each change is correct.\\
%\begin{table*}[!ht]
~\\
    \begin{tblr}{
			colspec = {X[3,t]X[c,t]},
			%cell{2}{1} = {c=2}{c}, % multicolumn
			%cell{1}{1} = {r=2}{c},
			rowsep = 3pt,
			hlines = 1pt,
			vlines = 1pt, % vlines can not pass through multicolumn cells
			%row{3-9} = {.5cm},
			%row{2-10} = {1.4cm}
		}
		%\hline
		\SetRow{lightgray,c} 
            Description  & (your answer)
            \\
        My project type was & \SetCell{orange,c} 
            % ===========   Replace the following line ============
            1
        \\

        My report (Sections 1-6) is this many pages long (for partial pages, use 0.3, 0.5, 0.7): & \SetCell{orange,c} 
            % ===========   Replace the following line ============
            4.5 
        \\

        My planning system was (add rows if you had more than one)  & \SetCell{orange,c} 
            % ===========   Replace the following line ============
            \textless replace: planning system \textgreater 
        \\
        My acting system was (add rows if you had more than one)  & \SetCell{orange,c} 
            % ===========   Replace the following line ============
            \textless replace: acting system \textgreater 
        \\

        {I invested approximately the following time, in hours, for each sprint: \\
          \gap (this is for the instructor to assess relative difficulty)}
        & \SetCell{black,c} \\
            \gap \gap \gap Sprint 1 : Development   & \SetCell{orange,c} 
                % ===========   Replace the following line ============
                \textless replace: integer [0,C] \textgreater  
            \\
            \gap \gap \gap Sprint 1 : Report Writing   & \SetCell{orange,c} 
                % ===========   Replace the following line ============
                \textless replace: integer [0,C] \textgreater  
            \\

            
            \gap \gap \gap Sprint 2 : Development & \SetCell{orange,c} 
                % ===========   Replace the following line ============
                \textless replace: integer [0,C] \textgreater  
            \\
            \gap \gap \gap Sprint 2 : Report Writing & \SetCell{orange,c} 
                % ===========   Replace the following line ============
                \textless replace: integer [0,C] \textgreater  
            \\

            \gap \gap \gap Sprint 3 : Development  & \SetCell{orange,c} 
                % ===========   Replace the following line ============
                \textless replace: integer [0,C] \textgreater  
            \\
            \gap \gap \gap Sprint 3 : Report Writing & \SetCell{orange,c} 
                % ===========   Replace the following line ============
                \textless replace: integer [0,C] \textgreater  
            \\
        
        I answered \underline{\phantom{some number}} research questions in my main report & \SetCell{orange,c} 
            % ===========   Replace the following line ============
            \textless replace: integer [0,C] \textgreater  
            \\

        I included  \underline{\phantom{some number}} plots in my main report & \SetCell{orange,c} 
            % ===========   Replace the following line ============
            \textless replace: integer [0,C] \textgreater  
            \\

        I wrote  \underline{\phantom{some number}} lines of code (excluding comments) for this project & \SetCell{orange,c} 
            % ===========   Replace the following line ============
            \textless replace: integer [0,C] \textgreater  
            \\
        
        (If included, these were optional) I added  an additional \underline{\phantom{some number}} plots in my appendix & \SetCell{orange,c} 
            % ===========   Replace the following line ============
            \textless replace: integer [0,C] \textgreater  
            \\
        
        In terms of difficulty compared to other semester projects I have done, I would rate this project as (1-5 scale with 1 being easiest 5 being most difficult)  & \SetCell{orange,c} 
            % ===========   Replace the following line ============
            \textless replace: \{ 1, 2, 3, 4, 5 \}  \textgreater
            \\

        In terms of what I learned, I would rate this (1-5 scale with 1 being "a little" and 5 being "a lot")& \SetCell{orange,c} 
            % ===========   Replace the following line ============
            \textless replace: \{ 1, 2, 3, 4, 5 \}  \textgreater  
            \\
        
    \end{tblr}
%\end{table*}
\vspace{2cm}
    \begin{center}
		\begin{minipage}{.11\linewidth}
			\Large{Name:}
			\vspace{1.75cm}
		\end{minipage}
		\begin{minipage}{.55\linewidth}
			\underline{REPLACE THIS TEXT BY TYPING YOUR NAME HERE}
            
			\vspace{.5cm}
            (For this report, typing your name here will suffice)\\
            ~\\
			I pledge on my honor that I have not given or received any unauthorized assistance on my programming project or report.
		\end{minipage}
	\end{center}

\clearpage

%\begin{table*}[!ht]
Instructor Assessment (Mak will fill this in)\\ 
    \begin{tblr}{
			colspec = {X[3,t]X[c,t]X[c,t]},
			%cell{2}{1} = {c=2}{c}, % multicolumn
			%cell{1}{1} = {r=2}{c},
			rowsep = 3pt,
			hlines = 1pt,
			vlines = 1pt, % vlines can not pass through multicolumn cells
			%row{3-9} = {.5cm},
			%row{2-10} = {1.4cm}
		}
		%\hline
		\SetRow{lightgray,c} 
            Approach  & Points Available & { Earned \\Instructor Only}  \\
        Clear technical approach &5 & \\
        Planning Environment (or other system) clearly explained & 5  &  \\
        Acting Environment (or other system) clearly explained & 5  & \\
        Includes Planning and Acting Component & 5 & \\

        \SetRow{lightgray,c} 
            Evaluation  & Points Available & Earned \\
        Clear exposition of claims, questions, variables and protocol & 10 & \\
        Source code and overall development work & 10 & \\
        Included baseline approach or ablation study, where appropriate (or demonstrated second order thinking/evaluation) & [5] & \\
        Evidence to support claims is discussed and included in Section~\ref{sec:figures-and-tables} & 10 & \\
        Discussion of results & 10 & \\
        
		\SetRow{lightgray,c} 
            Writing  & Points Available & Earned \\
        Clear story arc, paper within scope for project (2-4 pages of writing) & 10 & \\
        Followed checklist for placing plots and tables and overall structure; all plots are referenced & 5 & \\
        Grammar and overall writing is sound and citations are proper & 5 & \\
        
        \SetRow{lightgray,c} 
            Overall   & Points Available & Earned \\
            Technical Difficulty (related to project type) & 10 & \\
            Technical gains considering difficulty  & 10 & \\
            
        
        
    \end{tblr}
%\end{table*}

\twocolumn

\clearpage


%==========================================================
%==========================================================
\section{Introduction}

Within the rideshare domain, a very common pain point is connecting with the rider with their matched driver. This is especially true in scenarios where there are likely to be a very large number of riders searching for rides, such as at a concert or sporting event. In these scenarios, it is common for riders to be matched with drivers that are not in their immediate vicinity, and thus the rider must navigate to the driver. This can be a difficult task, especially if the rider is unfamiliar with the area or if there are obstacles in the way.

While much attention in AI planning has been focused on achieving the best possible planning performance and plan quality little attention has been given to leveraging the ability to simulate people as actors in the environment. In this project, we simulate a rideshare user attempting to find their assigned car in a parking lot. Rather than identifying the best possible plan to find the car, we investigate how different environments effect the ability of an agent to find their car when acting in the environment with a simple planning and acting algorithm. This allows us to ultimately draw conclusions about how different environment characteristics effect the ability of an agent to successfully find their car with the hope being that these insights can be used to design better human rideshare experiences in the future. As rideshare services move towards greater autonomy and uniformity in car design, it is important to maximize the efficiency of the particularly human intensive tasks.

%==========================================================
%==========================================================
\section{Background}

The motivation for this project builds on concepts introduced in earlier work that highlighted a gap in research on the final 100 yards of on-foot navigation within the broader last-mile problem~\cite{sametPILOTPilotingLast2024}. We acknowledge that the difficulties faced by riders in locating their car once in a general vicinity (ie parking lot) are especially unique as compared to simply walking to said area from an original starting location. Prior work has explored the challenges of on-foot navigation in urban environments, but there is a lack of research specifically addressing the nuances of the search task that is rideshare pickup.

The issue of locating a rideshare vehicle goes far beyond simple time efficiency savings. There have been numerous cases of riders entering the wrong vehicle and being physically harmed such as the case of Samantha Josephson who was murdered after entering a car she thought was her Uber. This case and others motivated the creation of Sami's law~\cite{smithSamisLaw2023} which requires a study to be done on the number of assaulted riders each year and the safety measures taken. In general there is very little regulation around rideshare safety within the US as compared to countries like China~\cite{stemlerDataPrivacyRegulation2025} which regulate where and how rideshare pickups can occur. Therefore the lessons learned from this project should not only be used to improve efficiency but also safety of rideshare pickups ensuring that riders can quickly and accurately and safely find their rideshare vehicle.

This project uses \RunLazyLookahead, \cite{ghallabActingPlanningLearning2025}, Algorithm~2.4. Briefly, this acting algorithm checks whether the agent has reached the goal. If an existing plan exists and \alg{Simulate} is not failure, it performs the next action in the plan. Otherwise it creates a new plan. 

% (Notice the use of macros from the `macro.tex` file -- these are optional but can really improve the readability of your report.)
%On the other hand, some of you created your own procedures or used something from the literature.  In that case, it might be a good idea to include pseudocode if you modified something from the book or created your own algorithm.  I don't really want you to worry about using the algorithmic environment.  For purposes of this report, I will accept psuedocode `lite', which is basically the text like we have been using all semester, similar to the following:
%\begin{scriptsize}
%\begin{verbatim}
%    procedure CrossPrint(arg1, arg2, arg3)
%    Q := { arg1 x arg2 }
%    while |Q| < arg3:
%        print This statement is false.
%        unused_item = pop(Q)
%\end{verbatim}
%\end{scriptsize}

%==========================================================
%==========================================================
\section{Approach}

%--------------------------------
\subsection{Planning Approach}

%--------------------------------
\subsection{Acting Approach}

%--------------------------------
\subsection{Environment}

Include details about your environment. 

If you use a URL link, you can put it in a footnote like this\footnote{\tinyurl{https://www.overleaf.com/learn/latex/}}  Otherwise, cite it as a reference like usual.

%==========================================================
%==========================================================
\section{Evaluation Plan}

Discuss the questions you asked and what you were expecting to see.

\paragraph{Independent variables}
Describe the independent variables you varied.

\paragraph{Dependent variables}
Describe the dependent variables you measured.

%==========================================================
%==========================================================
\section{Results}

Here, simply describe what you found without evaluating it.  You'll reference figures and tables from Section~\ref{sec:figures-and-tables}, as needed. 

%--------------------------------
\subsection{Discussion of Tradeoffs and Limitations}

Describe the key insights of your results.  Discuss limitations.


%==========================================================
%==========================================================
\section{What I learned}

Please describe here a few things you learned from this project.  

\begin{itemize}
    \item \textless replace with new item \textgreater
    \item \textless replace with new item \textgreater
    \item \textless replace with new item \textgreater
    \item \textless replace with new item \textgreater
\end{itemize}



\clearpage  %leave this command here so the tables and figures start on a new page
%==========================================================
%==========================================================
\section{Summary Figures and Tables}
\label{sec:figures-and-tables}

Here is where you will add figures and tables.  




%==========================================================
%==========================================================
\bibliography{aaai25}


%==========================================================
%==========================================================
%==========================================================
%==========================================================
\clearpage    %leave this command here so the appendices start on a new page
\appendix

\begin{center}
{\Large Appendix}
\end{center}

Should you need to, you can add sections here for additional content.

%==========================================================
%==========================================================
\section{Large Tables or Figures go here}

Table~\ref{tbl:example} shows an example table using the \textbackslash{table*} environment. 
Similarly, Figure~\ref{fig:example} shows an example large figure using the \textbackslash{figure*} environment.  

\begin{table*}
\centering
\begin{tabular}{ |p{3cm}||p{3cm}|p{3cm}|p{3cm}|  }
 \hline
 \multicolumn{4}{|c|}{Country List} \\
 \hline
 Country Name or Area Name& ISO ALPHA 2 Code &ISO ALPHA 3 Code&ISO numeric Code\\
 \hline
 Afghanistan   & AF    &AFG&   004\\
 Aland Islands&   AX  & ALA   &248\\
 Albania &AL & ALB&  008\\
 Algeria    &DZ & DZA&  012\\
 American Samoa&   AS  & ASM&016\\
 Andorra& AD  & AND   &020\\
 Angola& AO  & AGO&024\\
 Afghanistan   & AF    &AFG&   004\\
 Aland Islands&   AX  & ALA   &248\\
 Albania &AL & ALB&  008\\
 Algeria    &DZ & DZA&  012\\
 American Samoa&   AS  & ASM&016\\
 Andorra& AD  & AND   &020\\
 Angola& AO  & AGO&024\\
 Afghanistan   & AF    &AFG&   004\\
 Aland Islands&   AX  & ALA   &248\\
 Albania &AL & ALB&  008\\
 Algeria    &DZ & DZA&  012\\
 American Samoa&   AS  & ASM&016\\
 Andorra& AD  & AND   &020\\
 Angola& AO  & AGO&024\\
 Afghanistan   & AF    &AFG&   004\\
 Aland Islands&   AX  & ALA   &248\\
 Albania &AL & ALB&  008\\
 Algeria    &DZ & DZA&  012\\
 American Samoa&   AS  & ASM&016\\
 Andorra& AD  & AND   &020\\
 Angola& AO  & AGO&024\\
 \hline
\end{tabular}
\caption{Here is a caption for this example table, which came from the overleaf help documents at \tinyurl{https://www.overleaf.com/learn/latex/Tables}.  }
\label{tbl:example}
\end{table*}

\begin{figure*}
    \centering
    \includegraphics[width=0.5\linewidth]{figs/ssp.png}
    \caption{Here is a caption for this figure that you probably recognize. }
    \label{fig:example}
\end{figure*}

\end{document}
